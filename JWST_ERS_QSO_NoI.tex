%%%%%%%%%%%%%%%%%%%%%%%%%%%%%%%%%%%%%%%%%
% Professional Formal Letter
% LaTeX Template
% Version 1.0 (28/12/13)
%
% This template has been downloaded from:
% http://www.LaTeXTemplates.com
%
% Original author:
% Brian Moses (http://www.ms.uky.edu/~math/Resources/Templates/LaTeX/)
% with extensive modifications by Vel (vel@latextemplates.com)
%
% License:
% CC BY-NC-SA 3.0 (http://creativecommons.org/licenses/by-nc-sa/3.0/)
%
%%%%%%%%%%%%%%%%%%%%%%%%%%%%%%%%%%%%%%%%%

%----------------------------------------------------------------------------------------
%	PACKAGES AND OTHER DOCUMENT CONFIGURATIONS
%----------------------------------------------------------------------------------------

\documentclass[11pt,a4paper]{letter} % Specify the font size (10pt, 11pt and 12pt) and paper size (letterpaper, a4paper, etc)

\usepackage{graphicx} % Required for including pictures
\usepackage{microtype} % Improves typography
\usepackage{gfsdidot} % Use the GFS Didot font: http://www.tug.dk/FontCatalogue/gfsdidot/
\usepackage[T1]{fontenc} % Required for accented characters
%\usepackage[english,french]{babel}    
%\usepackage[english]{babel}    
\usepackage[margin=3.1cm]{geometry}
\usepackage[latin1]{inputenc}  
\usepackage{hyperref}


% Create a new command for the horizontal rule in the document which allows thickness specification
\makeatletter
\def\vhrulefill#1{\leavevmode\leaders\hrule\@height#1\hfill \kern\z@}
\makeatother

%----------------------------------------------------------------------------------------
%	DOCUMENT MARGINS
%----------------------------------------------------------------------------------------

\textwidth 6.75in
\textheight 9.750in  %9.25in
\oddsidemargin -.25in
\evensidemargin -.25in
\topmargin -1.75in %-1in
\longindentation 0.50\textwidth
\parindent 0.4in

%----------------------------------------------------------------------------------------
%	SENDER INFORMATION
%----------------------------------------------------------------------------------------

\def\Who{Nicholas P. Ross} % Your name
\def\What{, PhD} % Your title
\def\Where{Institute for Astronomy} % Your department/institution
\def\Where{University of Edinburgh} % Your department/institution
\def\Address{ Royal Observatory, Blackford Hill} % Your address
\def\CityZip{Edinburgh EH9 3HJ, U.K.} % Your city, zip code, country, etc

\def\Email{\href{mailto:npross@roe.ac.uk}{{\tt npross@roe.ac.uk}}} % Your email address
%\def\Email{npross@roe.ac.uk} % Your email address

\def\TEL{+44 (0)131-668 8351} % Your phone number

%\def\URL{\href{http://www.roe.ac.uk/~npross/Welcome.html}{http://www.roe.ac.uk/$\sim$npross/} % Your URL
\def\URL{\href{http://www.roe.ac.uk/~npross/}{http://www.roe.ac.uk/$\sim$npross}} % Your URL

%----------------------------------------------------------------------------------------
%	HEADER AND FROM ADDRESS STRUCTURE
%----------------------------------------------------------------------------------------

\address{
\includegraphics[width=1.2in]{University_of_Edinburgh_ceremonial_roundel.png} % Include the logo of your institution
%\includegraphics[width=1in]{avatar-logo-blueonwhite.png}
\hspace{5.1in} % Position of the institution logo, increase to move left, decrease to move right
\vskip -1.07in~\\ % Position of the text in relation to the institution logo, increase to move down, decrease to move up
\Large\hspace{1.5in}THE UNIVERSITY \hfill ~\\[0.05in] % First line of institution name, adjust hspace if your logo is wide
\hspace{1.5in}OF EDINBURGH \hfill \normalsize % Second line of institution name, adjust hspace if your logo is wide
%\makebox[0ex][r]{\bf \Who \What }\hspace{0.08in} % Print your name and title with a little whitespace to the right
%\makebox[0ex][r]{\bf \Who  }\hspace{0.08in} % Print your name and title with a little whitespace to the right
\makebox[0ex][r]{\bf }\hspace{0.08in} % Print your name and title with a little whitespace to the right
~\\[-0.11in] % Reduce the whitespace above the horizontal rule
\hspace{1.5in}\vhrulefill{1pt} \\ % Horizontal rule, adjust hspace if your logo is wide and \vhrulefill for the thickness of the rule
\hspace{\fill}\parbox[t]{2.85in}{ % Create a box for your details underneath the horizontal rule on the right
\footnotesize % Use a smaller font size for the details
\Who \What \\ \em % Your name, all text after this will be italicized
%\em
\Where\\ % Your department
\Address\\ % Your address
\CityZip\\ % Your city and zip code
\TEL\\ % Your phone number
\Email\\ % Your email address
\URL % Your URL
}
\hspace{-1.4in} % Horizontal position of this block, increase to move left, decrease to move right
\vspace{-1in} % Move the letter content up for a more compact look
}

%----------------------------------------------------------------------------------------
%	TO ADDRESS STRUCTURE
%----------------------------------------------------------------------------------------

\def\opening#1{\thispagestyle{empty}
{\centering\fromaddress \vspace{0.6in} \\                  % Print the header and from address here, add whitespace to move date down
\hspace*{\longindentation}\today\hspace*{\fill}\par} % Print today's date, remove \today to not display it
{\raggedright \toname \\ \toaddress \par}                  % Print the to name and address
\vspace{0.4in}                                                              % White space after the to address
\noindent #1                                                               % Print the opening line
% Uncomment the 4 lines below to print a footnote with custom text
%\def\thefootnote{}
%\def\footnoterule{\hrule}
%\footnotetext{\hspace*{\fill}{\footnotesize\em Footnote text}}
%\def\thefootnote{\arabic{footnote}}
}

%----------------------------------------------------------------------------------------
%	SIGNATURE STRUCTURE
%----------------------------------------------------------------------------------------

%\signature{\Who \What} % The signature is a combination of your name and title
\signature{\Who } % The signature is a combination of your name and title

\long\def\closing#1{
\vspace{0.1in} % Some whitespace after the letter content and before the signature
\noindent % Stop paragraph indentation
\hspace*{\longindentation} % Move the signature right
\parbox{\indentedwidth}{\raggedright
#1 % Print the signature text
%\vskip 0.25in % Whitespace between the signature text and your name
%\vskip 0.65in % Whitespace between the signature text and your name
\fromsig}} % Print your name and title

%----------------------------------------------------------------------------------------

\begin{document}

%----------------------------------------------------------------------------------------
%	TO ADDRESS
%----------------------------------------------------------------------------------------

\begin{letter}
{Space Telescope Science Institute\\ 
3700 San Martin Drive\\
Baltimore, MD 21218 \\
U.S.A.
}

%----------------------------------------------------------------------------------------
%	LETTER CONTENT
%----------------------------------------------------------------------------------------

%% -- an overview of the anticipated proposal, not to exceed 300 words, describing:
%%                ---  the proposed types of JWST observations and science goals,
%%                ---  how the proposed project supports the DD ERS goals and principles,

%\opening{Dear Sir or Madam,}
\opening{Dear STScI Director,}

\smallskip
%\noindent
\begin{center}
%{\bf Re: Notice of Intent for JWST Director's Discretionary Early Release Science program}
{\sc Revolutionary Quasar Science with {\it James Webb Space Telescope} Observations:\\
Notice of Intent for JWST Director's Discretionary Early Release Science program}
\end{center}

The two major sources off energy available to a galaxy are: nuclear
fusion and the gravitational potential energy associated with a
central super-massive black hole (SMBH). And, the link between massive
galaxies and the SMBHs that seem ubiquitous at their centers, is
assumed to be vital to the understanding of galaxy formation and
evolution. And, after dark matter, dust is the most mysterious
component of galaxies.

As such, in this Notice of Intent (NoI), we outline 6 particular
science cases that focus on the role of accreting supermassive black
holes and active galactic nuclei central engines, and their direct
consequences to galaxy formation and evolution at `Cosmic Dawn' and
`Cosmic Noon'.
%%
In particular, these science cases focus on:
\begin{itemize}
\item{Active Black Holes in very high redshift ($z>10$) galaxies;}
\item{Evidence for the Transition Mass that quenches AGN/quasar activity;}
\item{What physical processes trigger luminous QSO activity at Cosmic Noon?}
\item{Extremely Red Quasars, and placing them in an evolutionary context;}
\item{Changing Look AGN and Quasars, and understanding SMBH accretion and central engines;} 
\item{Discovering of true Type 2 QSOs at high redshift.}
\end{itemize}

The observational goals of the DD ERS program are to: {\it (i)} to perform a
$\sim$20 hour survey of {\it Spitzer} and WISE selected AGN and quasars; {\it (ii)} to
test multiple modes of JWST, in particular MIRI, and {\it (iii)} to dovetail
on the extragalactic deep field observations and produce timely and
enhanced science and data products for the community.

Noting the PIs host institution, we aim to test
multiple modes of the JWST MIRI including imaging, low-resolution
slitted and slitless spectroscopy, medium-resolution integral field
unit (IFU) spectroscopy and coronagraphy.

\closing{Sincerely,}

%----------------------------------------------------------------------------------------

\end{letter}
\end{document}
