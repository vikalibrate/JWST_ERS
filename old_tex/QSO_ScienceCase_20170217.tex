\documentclass[11pt,epsf]{article}

\input{format}

\setlength{\headheight}{15.2pt}
\setlength {\textwidth}{180mm} 
\setlength {\textheight}{240mm}
\topmargin=-25.0mm
\oddsidemargin=-10.00mm
\pagestyle{fancy}

%%
%%  Header and footer information
%%
\fancyhf{}
\renewcommand{\headrulewidth}{0pt} % remove lines as well
\lfoot{{\it JWST DD ERS QSOs}}
\setcounter{page}{1}
\cfoot{\thepage}
\rfoot{{\it Notice of Intent: Science Case}}

%%
%% ``Definitions'' go here
%%
\def\gsim{~\rlap{$>$}{\lower 1.0ex\hbox{$\sim$}}}
\def\lsim{~\rlap{$<$}{\lower 1.0ex\hbox{$\sim$}}}
\def \hmpc{\;h^{-1}{\rm Mpc}} 


\begin{document}

\vspace{-44pt} 
\begin{center}
  {\Large \bf Revolutionary Quasar Science with \\ 
    {\it James Webb Space Telescope} Observations:\\
    Notice of Intent for a Director's Discretionary \\
    Early Release Science Program\\}
\end{center}

\begin{quotation}
\noindent
{\it  The James Webb Space Telescope will revolutionize 
our understanding of the quasar phenomenon. 

}
\noindent
\end{quotation}


\section*{Overview}

%\smallskip
%\smallskip
%\noindent
%{\bf \underline{Introduction.}$\;$}
The link between massive galaxies and the central super-massive black
holes (SMBHs) that seem ubiquitous in them is now thought to be vital
to the understanding of galaxy formation and evolution.  As such, huge
observational and theoretical effort has been invested in trying to
measure and understand the physics involved in these enigmatic
systems.

\smallskip
\smallskip
\noindent
In this {\it Notice of Intent} (NoI), we outline six particular science
cases that focus on the role of accreting supermassive black holes and
active galactic nuclei central engines, and their direct consequences
to galaxy formation and evolution at Cosmic Dawn ($z>6$) and Cosmic
Noon ($z\sim2-5$).

\smallskip
\smallskip
\noindent
The key science goals for the DD ERS Program, are: 
\begin{itemize}
    \item{Active Black Holes in very high-$z$ galaxies.} 
    \item{Evidence for the Transition Mass that quenches AGN activity;}
    \item{What triggers luminous QSO acticity at Cosmic Noon?}
    \item{Extremely Red Quasars;}
    \item{Changing Look Quasars;}
    \item{True Type 2 QSOs are high redshift.}
\end{itemize}



\newpage
\medskip
\medskip

\section*{Science Case}

\smallskip
\smallskip
\noindent
\textbf{\textsc{I. Active Black Holes in very high-$z$ galaxies.}} 
JWST plans to image and obtain spectra of very high-$z$, $z\gtrsim12$
galaxies in legacy extragalactic deep fields (e.g. GOODS-N (Hubble
Deep Field North), GOODS‐S (Ultra Deep Field), Extended Groth Strip
(EGS), UKIDSS Ultra-Deep Survey (UDS) \ and COSMOS.
%%

\noindent
Discovery, and then population studies of the active AGN in the 
galaxies present in the deep fields. 

\noindent
Although AGN are not suspected to be the main contributors to 
Reionization, this still needs to be conclusively ruled out. 


\medskip
\medskip


\smallskip
\smallskip
\noindent
\textbf{\textsc{II. Evidence for the Transition Mass that quenches AGN activity.}} 
Recent high-resolution hydrodynamic simluations, e.g. the EAGLE
simulation, (Bower et al. 2017) suggest thar the Red Sequence and Blue
Cloud of galaxies is a consequence of the competition between star
formation-driven outflows and gas accretion on to the supermassive
black hole at the galaxy’s centre.  Moreover, at a stellar mass of
$\sim3{\times}10^{10}$ M$_{\odot}$ and a halo mass of $10^{12}$
M$_{\odot}$, the outflow ceases to be buoyant and star formation is
unable to prevent the build-up of gas in the central regions. This
triggers a strongly non-linear response from the black hole. Its
accretion rate rises rapidly, heating the galaxy’s corona, disrupting
the incoming supply of cool gas and starving the galaxy of the fuel
for star formation.


\medskip
\medskip


\begin{figure}[h]
  \begin{center}
   \hspace{-0.5cm}
    \includegraphics[height=8.5cm,width=12.2cm]
   {figs/Villforth_161106236v2_Fig4_right.png}
    \vspace{-10pt}
   \caption{Villforth et al, arXiv:1611.06236v2; their Figure 4. 
     Visual classification of all resolved AGN host galaxies and
     matched control galaxies. AGN are shown in red, control sample in
     green. The error bars show 1σ confidence intervals calculated
     following Cameron (2011).}
  \vspace{-14pt}
 \label{figtest-fig}
\end{center}
\end{figure}

\smallskip
\smallskip
\noindent
\textbf{\textsc{III. What triggers luminous QSO acticity at Cosmic Noon?}} 
Traditional thinking has long suggested that galaxy major mergers are responsible 
for luminous QSO activity. However, at the heigh of the quasar epoch, $z\sim2-4$, 
a.k.a. ``Cosmic Noon'', there is incredibly scant observational evidence that 
the majority of the black hole mass build up at the knee of the quasar luminosity 
function was trigger by a major merger process. 

\noindent
{\it JWST} will directly answer this outstanding question. 



\begin{figure}[h]
  \begin{center}
   \hspace{-0.5cm}
    \includegraphics[height=8.5cm,width=16.2cm]
   {figs/Glikman_2015_ApJ_806_218_Fig5_hr.jpg}
%    \vspace{-10pt}
   \caption{     \footnotesize
     {\it From: Glikman et al., 2015, ApJ, 806, 218; their Figure 5.}
     Two color HST images of the eight lower-redshift quasars studied in
     this paper imaged with F105W and F160W. Each row represents a separate
     object. The first column is the original image shown at a scale of 8′′
     × 8′′. The second column shows the residual image after subtracting
     only the point-source component. The third column shows the model for
     all but the point-source component; the blank frame is a source to
     which no host component could be fit. The final panel shows the full
     residual including masked regions and is indicative of the overall
     goodness of fit. Evidence of mergers and disrupted host galaxies is
     seen in most the sources. We apply the red–green–blue color-combining
     algorithm of Lupton et al. (2004) to our images, and we average the
     count rate from the F105W and F160W images to produce the green frame.
}
  \vspace{-14pt}
 \label{figtest-fig}
\end{center}
\end{figure}


\medskip
\medskip
\begin{wrapfigure}{r}{0.5\textwidth}
  \begin{center}
    \includegraphics[width=0.48\textwidth]{SDSS_J083448_ERQ.pdf}
  \end{center}
  \caption{\small 
    A WISE 3.4, 4.6 and 12$\mu {\rm m}$ image of a $z=2.59$ extremely red
    quasar, selected on its $r-[22]$ colour, discovered by [24].  This
    object has a 22$\mu$m flux indicative of $L_{IR} \gtrsim 10^{14}
    L_{\odot}$, and one interpretation could be we are witnessing the
    ``birth'' of an unobscured quasar.}
\end{wrapfigure}

\smallskip
\smallskip
\noindent
\textbf{\textsc{IV. Extremely Red Quasars}} 
The discovery of extremely red QSOs (ERQs) with $r~-~[22]~>~14$
colours from the WISE All-Sky Survey and spectroscopy from SDSS and
BOSS, seems to provide a key observational clue to the ``major
merger'' evolutionary theory for QSO activity.  However, the large
fraction of AGN which remain heavily obscured will need mid-infrared
spectroscopy in order to understand the role this optically hidden
population play in the evolution of galaxies and the integrated light
of the Universe. Given the fellowship timescale, this makes a natural
bridge to the {\it James Webb Space Telescope} and observations with
the MIRI spectrograph.


\medskip
\medskip

\smallskip
\smallskip
\noindent
\textbf{\textsc{V. Changing Look Quasars}} 
Recently ``Changing-look'' quasars (CLQs; [18-21]), have been
identified, and are defined to be luminous AGN which are observed to
have a dramatic appearance or disappearance of their broad
emission-line (BEL) component. CLQs are extremely important since they
offer a direct observational probe into the physical processes
dictating the structure of the AGN broad-line region (BLR), {\it and doing
so on observed-frame year-timescales.} The timescales that are measured for CLQs can potentially be associated with the viscous
timescale (associated with the drift time through the
accretion disk), the light crossing timescale (critical for
reverberation mapping and disk reprocessing) and the dynamical
timescale of the BLR.  {\it CLQs are thus an ideal laboratory for studying
accretion physics, as the entire system responds to a large change in
ionising flux on a human timescale}.


\medskip
\medskip

\smallskip
\smallskip
\noindent
\textbf{\textsc{VI. True Type 2 QSOs are high redshift.}} 
Proin non tempus velit. Etiam laoreet, enim nec scelerisque dictum,
tortor massa tempor enim, id pretium justo quam ac lectus. Maecenas
diam nibh, interdum at lobortis sit amet, dignissim et quam. Sed
tincidunt faucibus risus, congue tempus nisl consectetur
eget. Suspendisse venenatis turpis ut risus aliquam interdum. In at
velit sed ligula dictum dignissim ut et dui. Curabitur ac scelerisque
purus.

\noindent
Aliquam ac metus nec odio tempus pharetra sed nec diam. Sed eget arcu
nulla. Etiam elementum ultrices ligula, at iaculis libero feugiat
bibendum. Suspendisse potenti. Nam pharetra adipiscing
euismod. Quisque imperdiet dignissim odio, sed volutpat justo
tincidunt eu. Nunc vehicula pharetra suscipit. Integer aliquet pretium
ipsum vel ultrices. Nam rutrum nibh ac quam pulvinar molestie.


\newpage
\medskip
\medskip
\section{Science Case}

\smallskip
\smallskip
\noindent
\textbf{\textsc{Relevance to DD ERS.}} 
\underline{Survey Description.}\\

\smallskip
\smallskip
\noindent
\underline{Observation Modes.}\\
In particular, and noting the PIs host institution, we aim to test
multiple modes of the JWST MIRI including imaging, low-resolution
slitted and slitless spectroscopy, medium-resolution integral field
unit (IFU) spectroscopy and coronagraphy.

\smallskip
\smallskip
\noindent
{\underline Source of Targets.}\\
{\it (i):} The traditional Extragalactic Deep Fields; 
{\it (ii):} an


%\newpage

%%% 
%%%     \tiny    \scriptsize     \footnotesize     \small     \normalsize     \large    \Large     \LARGE     \huge     \Huge 
%%% 
\smallskip
\smallskip
\noindent
{\bf Mid-IR properties of QSOs and JWST.} 


\vspace{-16pt}
\begin{center}
%  {\Large \bf Figures and References \\}
 \medskip
 \medskip
 {\large \bf References}
    \vspace{-10pt}
\end{center}
\begin{multicols}{3}[]
\noindent
%\footnotesize
\scriptsize
%\tiny
\lbrack1\rbrack Fabian, 2012, ARAA, 50, 455 \\
\lbrack2\rbrack Alexander et al., 2012, NewAR, 56, 93\\
\lbrack3\rbrack Schneider et al. 2010, AJ, 139, 2360\\
\lbrack4\rbrack P\^{a}ris et al., 2012, A\&A, 548, A66 \\
\lbrack5\rbrack Dawson et al. 2013, AJ, 145, 10\\
\lbrack6\rbrack Ross et al., 2012, ApJS, 199, 3\\
%\lbrack7\rbrack Busca et al. 2013, A\&A, 522A, 96 \\
%\lbrack8\rbrack Slosar et al., 2013, JCAP, 04, 026 \\
%\lbrack9\rbrack Ross et al., 2013, ApJ, 773, 14, \\
%\lbrack10\rbrack Wright et al., 2010, AJ, 140, 1868\\
%\lbrack11\rbrack P\^{a}ris et al., 2013, A\&A, in prep. \\
%\lbrack12\rbrack Richards et al., 2006, ApJ, 131, 2766\\
%\lbrack13\rbrack Antonucci, 1993, ARA\&A, 31, 473\\
%\lbrack14\rbrack Urry \& Padovani, 1995, PASP, 107, 803\\
%\lbrack15\rbrack Sanders et al.\ 1988, ApJ, 325, 74\\
%\lbrack16\rbrack Canalizo\&Stockton, 2001, ApJ, 555, 719 \\
%\lbrack17\rbrack Hopkins et al., 2006, ApJS, 163, 1\\
%\lbrack18\rbrack White et al., 2012, MNRAS, 424, 933 \\
%\lbrack19\rbrack Geach et al., 2013, arXiv:1307.1706v1\\
%\lbrack20\rbrack Donoso et al., 2013, arXiv:1309.2277v1 \\
%\lbrack21\rbrack Finley et al., 2013, A\&A, accepted\\
%\lbrack22\rbrack Stern et al., 2012, ApJ, 753, 30\\
%\lbrack23\rbrack Assef et al., 2013, ApJ, 772, 26\\
%\lbrack24\rbrack Ross et al. 2013b, MNRAS in advanced prep.
%%Anderson et al. 2012, arXiv:1203.6594v1 \\
%%Banerji et al. 2012, arXiv:1210.6668v1\\
%Busca et al. 2012, arXiv: 1211.2616v1 \\
%Blake et al. 2011, MNRAS, 415, 2892 \\
%Cattaneo et al., 2009, Nature, 460, 213 \\
%Croom et al., 2005, MNRAS, 356, 415 \\
%Coppin et al., 2008, MNRAS, 389, 45 \\
%Croom et al., 2009, MNRAS, 399, 1755 \\
%da \^{A}ngela et al., 2008, MNRAS, 383, 565 \\
%Dawson et al.,  2012, arXiv1208.0022v1 \\ 
%%Dunkley et al., 2011, ApJ, 739, 52 \\
%Eisenstein et al., 2011,  AJ, 142, 72 \\
%Friemann et al. 2008, ARAA, 46, 385 \\
%Fiore et al.,  2011, arXiv1109.2888v1 \\
%Filiz Ak et al. 2012,  ApJ, 757, 114\\
%Haehnelt et al., 1994, MNRAS, 269, 199\\
%Hobbs et al., 2010, CQGra, 27h4013 \\
%Hopkins et al., 2007, ApJ, 654, 731 \\
%Hopkins et al., 2007b, ApJ, 662, 110 \\
%Lang et al., 2009, AJ, 137, 4400 \\
%Li et al, 2011, ApJ, 742, 33 \\
%Lidz et al. 2006, ApJ, 641, 41 \\
%Liu et al., 2011,  ApJ, 737, 101 \\
%Lonsdale et al., 2003, PASP, 115, 897\\
%Mauduit et al., 2012, PASP, 124. 714\\
%Palanque-Delabrouille~et~al., 2012, arXiv1209.3968v1 \\
%Palanque-Delabrouille~2012, arXiv1209.3968v1 \\
%P\^{a}ris, et al. 2012,  arXiv1210.5166v1\\
%Peth et al. 2011, AJ, 141, 105 \\
%Richards et al, 2006, AJ, 131, 2766 \\
%Ross et al., 2007, MNRAS, 381, 573  \\
%Ross et al., 2008, MNRAS, 387, 1323 \\
%Ross et al., 2009, ApJ, 697, 1634 \\
%Ross et al., 2012b, arXiv:1210.6389v1 \\
%Ross et al., 2012c, ApJL, in prep.\\
%Sanders et al. 2007, ApJS, 172, 86 \\ 
%Sawangwit~et~al.~2011,~arXiv:1108.1198v2 \\
%Schlegel et al., 2009, arXiv:0902.4680v1 \\
%Schneider et al., 2007,  AJ, 134, 102 \\
%Schneider et al., 2010,  AJ, 139, 2360 \\
%Seljak et al. 2009,  PhRvL, 091303\\
%Sesana et al., 2008, MNRAS, 390, 192 \\
%Simmons et al., 2011, ApJ, 734, 121 \\
%Shen et al. 2007, AJ, 133, 2222  \\
%Shen et al. 2009, ApJ, 697, 1656 \\
%Slosar et al, 2011, JCAP, 09, 1 \\
%Thorne, {\it Gravitational Radiation} 1987 \\
%White, astro-ph/0305474v1 \\

%Wang et al., 2009, ApJ, 697, L141 \\
%Wardlow et al., 2011, MNRAS, 415, 1479 \\
\end{multicols}

\iffalse
Science
===========================
— Hosts of z>2 QSOs
— SF properites of z>2 QSOs
— Mbh-Mbulge for z>2 QSOs
— Lots more here… 
— Lots more here… 


Technicals 
===========================
— MIRI spectroscopy (of PAHs??) 
— Rest UV/optical lines in which NIRCam/Spec filters (this is obviously easy, but still just needs to be written down! ;) 
— ERS is ~25hrs 
\fi




\end{document}
