\documentclass[11pt,a4paper]{article}
\input{format}

\begin{document}

   \title{JWST_MIRI_ETC_notes.texTITLE HERE}
  \author{AUTHOR HERE}
 \date{\today}
\maketitle


MIRI offers four different observing modes.

Imaging is available in nine photometric bands (5—28 μm, R=λ/Δλ∼5
), with a field-of-view (FOV) 74" x 113". The imager provides spatial resolution 0.18" at 5μm, and fully Nyquist samples the PSF for wavelengths ≥ 6.25 μm; the full width at half maximum is given by the expression 0.035 × λ μm (arc sec).  Detector subarrays allow for short exposure times in the case of bright targets and/or backgrounds. 



Coronagraphic imaging is facilitated by three four-quadrant phase masks at 10.65, 11.4 and 15.5 μm (24" x 24"), and a Lyot coronograph at 23 micron (30" x 30"). 

Low Resolution spectroscopy (R=λ/Δλ∼100
 at 7.5 μm) is possible at wavelengths 5—14 μm using either a 0.51" x 4.7" slit or a slitless prism. 

Medium Resolution integral field unit spectrosopy (R=λ/Δλ∼1550−3250
)  is implemented with four different IFUs, which span MIRI’s full wavelength range (5—28 μm) with fields of view ranging from 3.9" to 7.7".

\end{document}
